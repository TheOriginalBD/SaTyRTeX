\documentclass[a4paper,11pt]{article}

\usepackage{revy}
\usepackage[utf8]{inputenc}
\usepackage[T1]{fontenc}
\usepackage[danish]{babel}

\revyname{Matematikrevy}
\revyyear{2013}
% HUSK AT OPDATERE VERSIONSNUMMER
\version{1.0}
\eta{$3$ minutter}
\status{Færdig}
\responsible{Camilla}

\title{Ode Til Analyse 1}
\author{Freja}
\melody{Shout: Jeg Tror Det Kaldes Kærlighed}

\begin{document}
\maketitle

\begin{roles}
\role{S1}[Camilla] Sanger 1
\role{S2}[Karla] Sanger 2
\role{S3}[Helene] Sanger 3
\role{D1}[Cille] Danser
\role{D2}[Jasmin] Danser
\role{D3}[Miriam] Danser
\end{roles}

\begin{song}
\scene{Lys op, Børne-MGP stemning, bandet spiller Jeg tror det kaldes kærlighed med Shout}

\sings{S1} Hvordan gi’r vi mening til
Et uend’ligt summespil?
Tallene bli’r fler og fler
Tjek om delsum konverger’

\sings{S2} Sum af n’te del i p
Forholdstest og rottesten
Vi lærte om en dejlig ven
\sings{S1, S2, S3} Den hedder Grænsesammenligningstesten

\sings{S1, S2, S3} Denne sang den er til dig
Åh Jan Philip Solovej
Når funktionerne de ændrer sig
Så ta’r vi det skridt for skridt
Lært i analyse 1
Lad n gå mod uendeligt

\sings{S3} Nu blev faget mer’ abstrakt
Vi fandt en sum for pi eksakt
Lærte om vor tavlesvamp
\sings{S1, S2, S3} Valget tør og våd er… ikke mer’ en kamp 

\sings{S1, S2, S3} Denne sang den er til dig
Åh Jan Philip Solovej
Når funktionerne de ændrer sig
Så ta’r vi det skridt for skridt
Lært i analyse 1
Lad n gå mod uendeligt

\sings{S2} Er du nu lang langt væk eller så nær?
\sings{S1, S3} (Er du nu lang langt væk eller så nær?)

\sings{S3} Er metrikken 0, så du er her?

\sings{S1, S2, S3} Wowyeahhh

\sings{S1, S2, S3} Denne sang den er til dig
Åh Jan Philip Solovej
Når funktionerne de ændrer sig
Så ta’r vi det skridt for skridt
Lært i analyse 1
Lad n gå mod uendeligt

\sings{S1, S2, S3} Denne sang den er til dig
Åh Jan Philip Solovej
Når funktionerne de ændrer sig
Så ta’r vi det skridt for skridt
Lært i analyse 1
Lad n gå mod uendeligt

\sings{S1} Lad n gå mod uendeligt

\scene{Lys ned}
\end{song}
\end{document}