\documentclass[a4paper,11pt]{article}

\usepackage{revy}
\usepackage[utf8]{inputenc}
\usepackage[T1]{fontenc}
\usepackage[danish]{babel}

\revyname{Matematikrevy}
\revyyear{2013}
% HUSK AT OPDATERE VERSIONSNUMMER
\version{2.0}
\eta{$3$ minutter}
\status{Udkast}
\responsible{Camilla}

\title{På Science}
\author{Maling}
\melody{Simon and Garfunkle: Sound of Silence}

\begin{document}
\maketitle

\begin{roles}
\role{S1}[Simon] Sanger 1
\role{S2}[Helene] Sanger 2
\role{S3}[Peter] Sanger 3
\role{S4}[Camilla] Sanger 4
\role{S5}[NB] Sanger 5
\role{S6}[Karla] Sanger 6
\role{S7}[Agent A] Sanger 7
\role{S8}[Spectrum] Sanger 8
\role{S9}[Nine] Sanger 9
\role{S10}[Caro] Sanger 10
\end{roles}

\begin{song}

\sings{S1+S2} H.C. Ørsteds Institut
Det her vi regner uafbrudt
Vi nyder af fysikkens love
De seriøse og de sjove
Vi er nørder, og vi indrømmer det godt
Husk nu blot
At det er os på SCIENCE

\sings{S3+S4} Men indenfor på BioC
Der er der meget der kan ske
Der er travlt i lab’ratorier
Og lig’så vel i auditorier
DNA’en den er ikke hellig her
Som man ser
Så er det os på SCIENCE

\sings{S5+S6} Datalogisk Institut
Hvor ordet "PC" er forbudt
Her er mange datamater
Der er reklamer på plakater
For dataloger både tyk, såvel som tynd
Er det synd
Men vi er ogs' på SCIENCE

\sings{S7-10} Den nye bygning for Niels Bohr
Bli´r det der vi alle bor?
Hvis den ellers bliver færdig
Er der plads til hver en værdig
Vi får se - med dekanens nye plan
- - -
om det - bli'r os - fra SCIENCE

\sings{Alle} Og der er SaTyRrevy
samler gammel vel som ny
os - fra Sciences institutter
der hvor festen aldrig slutter
Og skulle det ske, at du scorer dig en yndig humanist
er det trist
For de... er ik'… fra SCIENCE

\end{song}
\end{document}