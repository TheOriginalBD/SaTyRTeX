\documentclass[a4paper,11pt]{article}

\usepackage{revy}
\usepackage[utf8]{inputenc}
\usepackage[T1]{fontenc}
\usepackage[danish]{babel}

\revyname{FysikRevy}
\revyyear{2014}
% HUSK AT OPDATERE VERSIONSNUMMER
\version{1.0}
\eta{$3$ minutter}
% Her skrives et estimat af sangens/sketchens varighed

\status{færdig}
% skriv færdig hvis sketchen er færdig, ellers skriv ideer

\title{Vil ha' 12}
%\author{Vektor}
\responsible{Ejnar, ejnarh@gmail.com}
\begin{document}
\maketitle

%Liste over roller og deres indehavere:
\begin{roles}
\role{S1}[Øsse] Sanger 1
\role{S2}[NB] Sanger 2
\role{S3}[Ejnar] Sanger 3
\role{F}[Shake?] Far
\role{M}[Nogen\texttrademark] Mor
\role{K}[Miriam] Kæreste
\end{roles}

%Liste over rekvisitter. Behold teksten 'Person, der skaffer',
%indtil det er sikkert, hvem der skal have ansvaret for rekvisitten
\begin{props}
\prop{1 Stol}[Person, der skaffer]
\prop{ca. 10 tøjbøjler}[Person, der skaffer]
\prop{Stang som bøjlerne kan hænge på når man trækker bagtæppet lidt til side.}
\prop{Vasketøjskurv}[Person, der skaffer]
\prop{Tallerken med indtørrede lasagnerester}[Person, der skaffer]
\prop{Opvaskebørste}[Person, der skaffer]
\prop{Euklidbog (normal - ikke bygget)}[Person, der skaffer]
\end{props}

\begin{sketch}

\scene Lys op. S1 sidder p?en stol
%Brug \scene inden alt scenespil inkl. lys og lyd fra TeXnikken.

\scene{Slow jazz, spoken word}

\says{S1} Mit fag started i denne uge, og jeg vil ha tolv.
\says{S1} Så oplæsning begynder nu, for jeg vil ha tolv.
\says{S1} Og jeg vil bruge alt min tid på dette fa'ag
\says{S1} for jeg vil ha tolv x3

\scene{Musik starter, S2 og S3 kommer ind på de to første linier}

\says{S2} jeg har ikke mere rent tøj, for jeg vil ha tolv.
\says{S2} og mit køkken ligner møg, for jeg vil ha tolv.
\says{S2} og jeg har sovet på min stol i flere uger
\says{S1,S3} I flere uger?!!!
\says{S2} for jeg vil ha tolv x3


\says{S3} Nu har jeg mistet mit job, for jeg vil ha tolv.
\says{S1} Lever af rester jeg samler op, for jeg vil ha tolv.
\says{S2} Nu bor jeg på HCØ på 4. sal
\says{S1} Hos Qdev
\says{Alle} for jeg vil ha tolv x3

\says{S1} Men du har jo ikke været på caféen? i flere uger!
\says{S2} for jeg vil ha tolv
\says{S2} og du vil heller ikke være med til at ryge dig høj!
\says{S3} for jeg vil ha tolv.
\says{S3} Og du var jo ikke engang inde og se SATYRrevy!
\says{S1} Jeg vil ha tolv
\says{Alle} for jeg vil ha tolv x3

\act{F+M kommer ind og går sørgmodigt over scenen.}
\says{S3} Jeg har ik' set min mor og far, for jeg vil ha tolv.
\act{K kommer ind, giver S1 en lussing og skrider.}
\says{S1} og min kærste skred i en fart, for jeg vil ha tolv.
\act{S3 trækker bagtæppet et stykke fra. Der er ingenting. Han trækker det for igen.}
\says{S2} og jeg har nu ikke længere nogen venner
\says{S1} ha, taber!
\says{S3} for jeg vil ha tolv x3

\act{S2 sætter sig på stolen og hiver Euklidbog frem.}
\says{S2} Sidder og læser Euklid, for jeg vil ha tolv
\says{S2} Hun sir' jeg spilder min tid, men jeg vil ha tolv.
\says{S2} Så tilbød hun mig sex, jeg sagde neeej
\says{S1,S3} Hvar?!Hvad?!Hvorfor?
\says{S2} for jeg vil ha tolv x3

\says{S3} Ved ikk' hvad dag det er, for jeg vil ha tolv.
\says{S1} og jeg glemte eksamen var her, for jeg vil ha tolv.
\says{S2} og nu har jeg istedet for fået minus tre
\says{S1,S3} Hvad ikke tolv?
\says{S2} nej jeg vil ha tolv
\says{alle} Vi vil ha tolv x2




\scene Lys ned

\end{sketch}
\end{document}
