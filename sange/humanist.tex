\documentclass[a4paper,11pt]{article}

\usepackage{revy}
\usepackage[utf8]{inputenc}
\usepackage[T1]{fontenc}
\usepackage[danish]{babel}


\revyname{DIKUrevy}
\revyyear{2013}
\version{1.0}
\eta{$3.75$ minutter}
\status{Færdig}
\responsible{Spectrum}

\title{Humanist}
\author{Sebastian Paaske Tørholm}
\melody{Dario von Slutty: ``Onani''} % http://youtu.be/vNBIIXZJJ-k

\begin{document}
\maketitle

\begin{roles}
\role{K}[Spectrum] Karl von Koder
\role{B1}[Joakim] Backupsanger 1
\role{P1}[Nine] Pige 1
\role{D1}[Caro] Dreng 1
\role{D2}[Freja] Dreng 2
\role{D3}[Sara] Dreng 3
\end{roles}

\begin{song}
    \scene{K, B1 og B2 starter på scenen}
    \sings{B1+B2} {
        Humanist - en KUAist
        Humanist - en KUAist
        Humanist - det’ godt nok trist:
        Det ta’r dem 10 semestre
        og kan ikke brug’s til sidst.
    }
%
    \sings{K} {
        Nu skal I høre:
        Jeg er et meget gammelt røvhul; jeg har ECTS-behov,
        men kurserne jeg tager, de gi’r mig aldrig lov.
        De taler mig i søvne her til forelæsningen
        Og når den så er ovre, har jeg det hele glemt igen.

        Hvorfor tage kurser der altid dumper mig?
        Det er derfor at jeg siger: Der er en nem're vej:
    }
    \sings{K} {
        Humanist - en KUAist
        Humanist - en KUAist
        Humanist - det’ godt nok trist:
        Det ta’r os 10 semestre
        og kan ikke brug’s til sidst.
    }
    \sings{B1+B2} {
        Humanist - en KUAist
        Humanist - en KUAist
        Humanist - det’ godt nok trist:
        Det ta’r dem 10 semestre
        og kan ikke brug’s til sidst.
    }
%
    \sings{K} {
        Se, de fleste går og drømmer om at arbejde med kode,
        men jeg vil bar’ være færdig -- arbejdsløshed er på mode!
        Jeg er ligeglad med penge; jeg har børneopsparing
        Og når den så løber tør kan jeg pizza udbring’

        Hvorfor sku' jeg arbejd’ og ikke ha’ tid til leg?
        Det er derfor at jeg siger: Der er en nem're vej:
    }
    \sings{K} {
        Humanist - en KUAist
        Humanist - en KUAist
        Humanist - det’ godt nok trist:
        Det ta’r os 10 semestre
        og kan ikke brug’s til sidst.
    }
    \sings{B1+B2} {
        Humanist - en KUAist
        Humanist - en KUAist
        Humanist - det’ godt nok trist:
        Det ta’r dem 10 semestre
        og kan ikke brug’s til sidst.
    }
%
    \sings{K} {
        Så er det ovre; jeg er endelig blevet fri
        Jeg er blevet kandidat i østrigsk eskimologi
        Der findes intet job hvori denne grad kan bruges
        Så alle jobsamtaler sluttes af med at der buh’es.

        Her er det så jeg fortryder jeg droppede ud
        Hvis bare jeg kunne kode så fremtiden lysere ud
    }
    \sings{K} {
        Humanist - en KUAist
        Humanist - en KUAist
        Humanist - det’ godt nok trist:
        Det ta’r os 10 semestre
        og kan ikke brug’s til sidst.
    }

    \scene{D træder ind på scenen}

    \sings{B1+B2} {
        Datalog - han er jo god
        Datalog - han er jo god
        Datalog - får job, værsgo’
        Det ta’r dem 20 semestre
        men i det mindste kan de kode.
    }

    \sings{B1+B2} {
        Datalog - han er jo god
        Datalog - han er jo god
        Datalog - får job, værsgo’
        Det ta’r dem 20 semestre
        men i det mindste kan de kode.
        Det ta’r dem 20 semestre
        men i det mindste kan de kode.
    }
\end{song}

\end{document}

