\documentclass[a4paper,11pt]{article}

\usepackage{revy}
\usepackage[utf8]{inputenc}
\usepackage[T1]{fontenc}
\usepackage[danish]{babel}

\revyname{FysikRevy}
\revyyear{2011}
\version{1.0}
\eta{$2$ minutter}
\status{Færdig}
\responsible{Camilla}

\title{Hvis pi'r var som fysik}
\melody{Ørkenens Sønner: ``Lem, kære lem''}

\begin{document}
\maketitle


\begin{roles}
\role{S}[Camilla] Sanger
\role{S}[Peter] Sanger
\role{S}[Nine] Sanger
\role{S}[Agent A.] Sanger
\role{S}[Simon] Sanger
\role{S}[Øsse] Sanger
\end{roles}

\begin{props}
    \prop{Briks}[Person, der skaffer]
    \prop{Lagen}[Person, der skaffer]
\end{props}

\begin{song}

\sings{S} Jeg ønsker tit
At pi'r var som fysik
Logiske, simple, beregnelige
Kæresten ku' være en bølgefunktion
Sidespring forklares
Som en superposition
(og) At slå op bli'r let, hvis blot man bru-u-u'r
Annihilationsoperator'
Oh yeah!

\sings{S} Verden var skøn
hvis pi'r var som fysik
Kærlighed let
som kvantemekanik
Når man sku' score
Så fandt man blot den pi'
(hvor) scoreoperatoren
(den) havde størst egenværdi

\sings{S} Ja, verden var skøn
Hvis sex det var diskret
kontinuert går galt - helt fatalt

\sings{S} Og forholdsdiskussioner
blir matrixoperationer
Hvis en dag at pi'r
blir som fysik
Simpel logik!
\end{song}

\end{document}