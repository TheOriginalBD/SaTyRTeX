\documentclass[a4paper,11pt]{article}

\usepackage{revy}
\usepackage[utf8]{inputenc}
\usepackage[T1]{fontenc}
\usepackage[danish]{babel}

\revyname{FysikRevy}
\revyyear{2014}
% HUSK AT OPDATERE VERSIONSNUMMER
\version{4.2}
\eta{$3$ minutter}
% Her skrives et estimat af sangens/sketchens varighed

\status{færdig}
% skriv færdig hvis sketchen er færdig, ellers skriv ideer

\title{Frisør}
%\author{Emil Machine}

\begin{document}
\maketitle

%Liste over roller og deres indehavere:

\responsible{<Sidsel>}

\begin{roles}
\role{F}[Freja] Fysiker
\role{H}[Sidsel] Hårfrisør
\end{roles}

%Liste over rekvisitter. Behold teksten 'Person, der skaffer',
%indtil det er sikkert, hvem der skal have ansvaret for rekvisitten
\begin{props}
\prop{Frisørstol, der kan dreje}[Person, der skaffer]
\prop{lille bord}[Person, der skaffer]
\prop{Frisørsaks}[Person, der skaffer]
\prop{Kam}[Person, der skaffer]
\prop{Frisørkappe}[Person, der skaffer]
\prop{Frisørbælte}[Person, der skaffer]
\prop{frisørgøjl}[skaffer vi selv]
\end{props}

\begin{sketch}

\scene{Til Tekik: Ønsker head-sets til begge roller. Vigtigt, da det er essentielt for skecthen at begge roller har begge hænder fri.}

\scene{Lys op}

\scene{F sidder i en frisørstol med front mod publikum. H står bagved og klipper
F's hår.}

\says{H} ...hun var bare en kæmpe kælling. Og så sagde jeg: Mor, sådan skal du bare ikke snakke til mig. Og så har jeg ikke set hende side. [Pause] Nå, hvad laver du så?

\says{F} Jeg skriver phd i
kvantemagnetohydromikromakrooptofeltkernekaos... Hvad laver du?

\says{H} Jamen, jeg er frisør!

\says{F} [almindeligt interesseret] Frisør. \act{Får øje på saksen i H's hånd} [entusiastisk] Ej! Er det en SAKS? Er der ikke vildt meget saks i at være frisør?

\says{H} Joh, det er der vel.

\says{F} Jeg har aldrig helt forstået begrebet saks. Man har saks og hår, og så
sker der et eller andet, og så er ens hår blevet kortere!

\says{H} Nåmen, det kan jeg godt forklare! Så, saks er lige som.. okay, du
kender godt knive ikke?

\says{F} Jo! Jeg kender knive!

\says{H} Skide godt! Så forestil dig du har én kniv. \act{går hen og tager læbestift på. Kan ikke se F}

\says{F} Én kniv! \act{mimer en kniv}

\says{H} Okay, så forestil dig at du har to knive.

\says{F} To knive! \act{mimer én kniv i hver hånd}

\says{H} Så forestil dig, at du skærer med dem samtidig.

\says{F} \act{skær med begge knive ud i luften} Samtidig. Okay, det tror jeg godt at jeg forstår! Så timingen er rigtig vigtig? Så de ikke kommer ud af fase? \act{Skærer ude af fase}

\says{H} \act{Er ovre ved bord. Kan stadig ikke se F} Neaj, joh, nej.. det kommer helt af sig selv.

\says{F} \act{skærer med de to knive ude af takt} Af sig selv? Det tror jeg ikke, jeg helt
forstår. Men jeg lærte heller aldrig saks i folkeskolen!

\act{H går tilbage til F}

\says{H} Der var da ikke noget fag om saks i folkeskolen.

\says{F} Var der ikke engang et fag om det?! Så er det jo KLART at jeg ikke kan
finde ud af det!

\says{H} Altså, jeg tror godt du kan finde ud af det, hvis du sådan rigtig
prøver. Det er ligesom hvis du har saks og papir.

\says{F} Wow! Både saks OG papir! Du må være helt vildt klog!

\says{H} Neej! Eller jo, eller. Prøv at hør, det er jo ikke kvantefysik det her!

\says{F} Nej, det her svært! Du får det til at lyde så simpelt!

\says{H} Det ER simpelt!\act{klipper demonstrativt med saksen foran F}

\says{F} Hah, nørd. \act{Får øje på kammen i H's anden hånd, og tager den ud af hånden på
ham og rækker den op prisende op i luften mens F siger:} Ej, er det en KAM?!

\scene lys ned.

\end{sketch}
\end{document}