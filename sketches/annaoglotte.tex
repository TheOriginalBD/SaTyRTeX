\documentclass[a4paper,11pt]{article}

\usepackage{revy}
\usepackage[utf8]{inputenc}
\usepackage[T1]{fontenc}
\usepackage[danish]{babel}

\revyname{SatyrRevy}
\revyyear{2014}
% HUSK AT OPDATERE VERSIONSNUMMER
\version{1.1}
\eta{3 minutter}
\status{færdig}
\responsible{Heine}

\title{Anna og Lotte i lab}
\author{MolBioKemRevyen 2014}

\begin{document}
\maketitle

\begin{roles}
    \role{A}[Heine] Anna
    \role{L}[Jakob] Lotte
    \role{K}[NB] Kirsten
\end{roles}

\begin{props}
    \prop{Geo dukke}[person der skaffer] (med boller der kan klippes af)
    \prop{Syre}[person der skaffer]
    \prop{Forsøgsvejledning*2(A4 ark)}[person der skaffer]
    \prop{Bombekaloriemeter}[person der skaffer] (Stort nok til at side på)
    \prop{Saks}[person der skaffer]
    \prop{Lab-ting}[person der skaffer]
\end{props}


\begin{sketch}
\scene Microfoner: Headsets til alle 3. \\ Audio: lyd que* lige efter 'bare rolig, vi er også færdige'

\scene{Scene: Bord med lab-ting står på bord, som står fremme på scenen, Og bambekalorimeteret starter også inde. Alle skuespillere starter også inde.}
\scene{Lys op.}

\says{K} Hej Anna, hej Lotte,

\says{A \& L}[Samtidigt] Hej Kirsten

\says{K} I dag skal vi (HOST HOST HOST --> dyb stemme) lave Fysisk kemi forsøg!

\says{L} Øv, jeg troede vi skulle bage boller!

\says{A} Ja, jeg havde lige glædet mig til boller, Kirsten.

\says{K} Nej, det har vi snakket om: Ingen boller i dag.

\says{A} [plagende] Kirsten, Kirsten, Kirsten ... Må Geo også være med til psykologi-forsøg?

\says{K} Nej Anna; Fysisk kemi forsøg: Vi skal finde koge- og frysepunkter for vand. Og selvfølgelig må Geo være med.


\says{L} Ja, og så sagde vi at Geo var termometer. 

\says{K} Nej, Lotte. Man kan altså ikke bruge Geo som termometer

\act{A Putter Geo ned i flamingo/gryde/labudstyr}

\says{A} Du siger bare selv stop, ik’ også Geo?

\says{K} Altså unger, nu skal I være søde mod Geo.

\says{A} Det er vi da også Kirsten, vi leger bare med ham.

\act{A holder Geo, mens Lotte skal til at hælde syre ud over ham}

\says{K}NEJ Lotte! Der er slet ikke syre med i forsøget! \act{Tager syren væk fra Lotte} 

\says{K}Se jeg har lavet en fin øvelsesvejledning. Dem skal vi følge.

\act{A og Lotte modtager vejledningen}

\says{L} Er det det samme som at male inden for stregerne Kirsten?

\act{A \& L lægger vejledningerne på bordet}

\says{K } Ja det kan man godt sige. Lad os prøve at begynde. Det første vi skal huske er, at vores udstyr skal være HELT sterilt.

\says{L} Kirsten siger du skal af med bollerne, Geo!

\says{K} Nej børn, det har vi snakket om. - ingen boller i dag!

\says{A\& L} [i kor] Jaaaa! Ingen boller!



\act{L klipper bollerne af Geo}

\says{K} Hvor er det dejligt I endelig hører efter børn. ...Nå men vi skal videre. her står at vi skal tilsætte urea.

\says{L} Kom Anna, vildledningen siger, du skal tisse på Geo

\act{A forsøger at tisse på geo}

\says{A} Kirsten. Kirsten! Hvad skal man gøre hvis man ikke skal tisse?

\says{K} Anna! Det HAR vi snakket om. Man ikke tisser på Geo.

\says{A} Jamen bar’ en lille smule…

\act{K tager Geo fra A og lægger den på bordet}

\says{L} \act{ser bombekalorimeteret} Hvad er det her, Kirsten?

\says{K} Det er et bombekalorimeter.

\act{A tager Geo, og begynder at putte ham ned i bombekilometeren}

\says{L} ÅRH! Er det sådan et som siger kæmpe bang og bomber kalorier?

\says{K} Nej det siger ikke en lyd. Man bruger et bombekalorimeter til måle forbrænding af
forskellige stoffer. Fx. slik.

\says{A \& Lotte}[Kikker på hinanden] mmmmmh…

\says{A} Se lige mig Kirsten! Jeg bruger det der bombe-kilometer \act{Putter Geo ind i bombekal.*}


\says{K}[MEGET chokeret] Stop, børn! Hvad er det I gør? Det må I ikke!


\says{L} Bare rolig Kirsten, vi er også færdige.*

\scene{Bib/pling/microbølgeown-færdig-lyd}

\act{A, Lotte og Kirsten kigger sammen ned i kalorimeteret. Anna og Lotte ser meget spændte ud,
mens Kirsten er meget bekymret.}

\act{A og L kigger fortvivlet op}

\says{A} Kirsten? Hvor er Geo henne?

\says{K}[grædefærdig] Anna, Geo er væk… Geo er død…

\says{L} ....Eeer Geo død..?

\says{A} Jamen, jamen, jamen Kirsten…

\says{L} …Hvem skal nu skrive vores rapport?

\scene{Lys ned.}

\end{sketch}
\end{document}
