\documentclass[a4paper,12pt]{article}

\usepackage{revy}
\usepackage[utf8]{inputenc}
\usepackage[T1]{fontenc}
\usepackage[danish]{babel}
\pdfoptionpdfminorversion 4

\revyname{Biorevy}
\revyyear{2013}
% HUSK AT OPDATERE VERSIONSNUMMER
% UNDLAD AT SKRIVE I TEMPLATE.TEX - KOPIÉR OG OMDØB I STEDET FOR
\version{0}
\eta{$10$ minutter}
\status{Færdig}

\title{Antropologisk Have}
\author{Biorevy}

\begin{document}
\maketitle


%\begin{texxers}
%	\texxer{Navn}[email@exmpl.com]
%\end{texxers}


\begin{roles}
	\role{AntJ}[Joakim]Antropolog Jesper
	\role{Ba1}[Steen]Barn 1
	\role{Ba1}[Julie]Barn 2
	\role{Fy1}[Heine]Fysiker
	\role{Fy2}[Anna]Fysiker
	\role{Bj}[Daniel]Fysikerdyrepasser-Bjarne
	\role{CBS}[Shake]CBS'er
	\role{Dat1}[Asger]Datalog 1
	\role{MB}[Caro]Molbo
	\role{Pæd1}[Olivia]Pædagog 1 (Pet'a gog)
	\role{Pæd2}[Magnus]Pædagog 2
\end{roles}


\begin{props}
	\prop{Anlæg}[Hvem skaffer?]2 "dyre"anlæg - 1 på hver side af scenen. Dvs. to indhegninger, evt. bare med snore, muligvis kan man nøjes med et.
	\prop{FeZ}[Hvem skaffer?] Fez én masse
	\prop{kitler}[Hvem skaffer?] Kitler
	\prop{kokosnødder}[Hvem skaffer?] 2 kokosnødder
	\prop{Skovl}[Hvem skaffer?] Muge-udstyr, skovl.
	\prop{Jakkesæt}[Hvem skaffer?] jakkesæt til CBS
	\prop{Hule}[Hvem skaffer?] pengeskabshule til at stå over lemmen(CBS'erens hule)
	\prop{Kittel}[Hvem skaffer?] Molbo-kittel med ekstra arme
	\prop{pædagogtøj}[Hvem skaffer?] pædagogtøj
	\prop{guitar}[Hvem skaffer?] En akustisk guitar.
	\prop{Laptops}[Hvem skaffer?] Computere til dataloger.
	\prop{Cola}[Hvem skaffer?] Cola til dataloger.
	\prop{Chips}[Hvem skaffer?] Chips til dataloger.
\end{props}

\begin{sketch}

\textbf{Bemærkninger til Teknikken}:Dette er en stor sketch med mange situationer; der skal være lys skiftevis på hver side.\\ For det meste "normalt" dagslys. Måske grønt/blåligt på CBS'eren, Rødt scenelys til allersidst. Folk der styrter rundt i salen må der nok gerne være spot på.
Til sidst, efter replikken ``Husk Guldøl!'' skal salslyset tændes.


\scene Lys op. Antropolog 1 står alene på scenen i venstre side med et par børn omkring sig - sidder evt forrest på scenen med ryggen til publikum.

\says{AntJ} Hej børn, jeg hedder Pia, og jeg er antropolog. I dag skal I på rundvisning i Antropologisk Have! I skal i dag se nogle af de mest karakteristiske menneskeformer, som vi har udstillet her. Lad os straks komme i gang!

\act{Antropolog 1 går en lille tur på scenen. På højre side af scenen kommer fysiker anlægget op.}

\scene Aber: 2 Fysikere i et anlæg med en dyrepasser. Fysikerne sidder på række og piller lopper ud af hinandens lange hår. Dyrepasser muger fez'er ud.
\says{AntJ} Ja, her er vores strålende bestand af flotte fysikere! Og her er Bjarne, han passer fysikerne i dag. Hej Bjarne.
\says {Bjarne} Hej børn
\says{AntJ}Fortæl os lidt om disse langhårede individer.
\says{Bjarne}Ja, de er jo en meget energisk individtype, men yderst svær at holde i fangenskab. De skal udfordres dagligt, derfor har vi givet dem en uløselig knude af super-strenge. (til fysiker) Ja, det er svært, hva?
\scene En fysiker tager to kokosnødder og prøver at smadre dem sammen.
\says{Bjarne}Hov se, der er en der prøver at lave kernefusion -- det må den ikke.
\scene Bjarne tager kokosnødderne fra fysikeren. Fysikerne bliver sure, og begynder at kaste med fez ud på publikum og Bjarne og antropologen. Antropolog bliver jaget ud.

\says{AntJ}Ej, nu kaster de med fez igen.

\scene Antropolog og børn går over på den anden side af scenen hvor der står en skoletjeneste-gut.

\says{AntJ}Børn, det her er Preben, han er også antropolog. Han vil vise jer rundt resten af dagen.
\says{AntJ}Hej børn. Ja, i dag skal vi se på et meget spændende eksemplar fra CBS. Mange er bange for dem, men i virkeligheden er den ikke særlig farlig.

\says{AntJ}Vi har lavet en hule til ham, som han kan gemme sig i. Lad os prøve at lokke ham frem.

\scene CBS'eren bor nede under lemmen med lille hule (fx et pengeskab) på toppen. CBS'eren sidder i jakkesæt og gemmer sig - lidt vampyr-agtig (blodsugende). AntJ lokker ham frem ved at brede Børsen ud på gulvet, have penge på en snor (eller på et serveringsfad), og hviske "insiderhandel". CBS'eren kommer krybende ud.

\says{AntJ}[Henvendt til børnene]Bare rolig, den gør ikke noget når den lige har fået penge. I må gerne klappe ham.
\act{Det ene barn er meget nervøst, men tager mod til sig og klapper ham.}
\says{Barn1}Ej, han er helt kold!
\act{Barn 2 går amok i klapning, meget overstadig.}
\says{AntJ}[til barn 2]Du er godt nok ikke bange af dig, hva'?
\says{Barn2}Nej, min far arbejder med dem, han er nemlig fra SKAT.
\act{Barnet stråler af stolthed, CBS'er hvæser og trækker tilbage i sin hule ved ordet SKAT.}

\says{AntJ}Hov, den gar under jorden når man snakker om SKAT. Så går vi videre\ldots

\scene Mens de går rundt på scenen hiver barn1 antropologen i ærmet for at få hans opmærksomhed.
\says{Barn1}Øhm, Hr, øhm, Antromopol? Jeg har hørt der er sådan en mærkelig en der er vågen om natten og det ligner lidt den der gollum fra ringenes herre men bare lidt klammere\ldots og måske lidt mere lys-sky? Må vi godt se det? Kan vi kan vi kan vi??? \emph{Pleeease?}
\says{AntJ}Jo, jeg tror godt jeg kender den du snakker om. Vi er faktisk lige i nærheden. Det er de truede dataloger i buret herovre.

\scene Lys op på dataloger. Noget med computere og chips og cola.
\says{AntJ}Datalogerne er truede fordi de har en sprogbarriere -- de taler kun i koder --
\act{Datalog taler i koder}
\says{AntJ}Derfor har de svært ved at formere sig i naturen. For at genoprette den naturlige bestand, har vi startet et avlsprogram.
\scene Dataloger fortsætter med at tale i koder - børnene er meget interesserede i dem, prøver at røre ved dem, datalogerne er sky.
\says{AntJ}Vi skal nu til at avle dem med de elitestuderende molboer, for en datalog med 8 arme skal man ikke kimse af. Og I er rigtig heldige, for netop nu skal første forsøg afprøves.
\scene En molbo i kittel træder frem og løfter armene. Hver arm har forskelligt - pipetter, artikler, petriskåle?. Hun går hen til datalogen, hiver ham om bag scenetæppet. Blod og tøj flyver ud på scenen hvis muligt.
\says {AntJ} Først parrede vi datalogen med matematikeren, men der endte vi op med infertile datamatikere.
\says{AntJ}(henvendt til publikum) Hovsa, det var måske ikke helt børnevenligt\ldots
\says{AntJ}Så børn, lad os gå over til børneafdelingen.

\scene Lys op på pædagoger. Her er en meget lav indhegning hvor der sidder nogle pædagoger med kaffekopper, nemt at komme ind til dem.
\says{AntJ}Her kan man komme ind og blive klappet af pædagogerne. I får lige noget kaffe at give dem, så bliver de så glade.
\scene Pædagogerne går rundt og klapper på børn. Børnene får lidt kaffe fra AntJ's thermokande og kan give det til pædagogerne. Da pædagogerne smager begynder de at blive ophidsede.
\says{AntJ}Hov hov hov. Ro i buret, hvad sker der med jer?
\scene Pædagogerne ignorerer ham.
\says{AntJ}[Kigger på sin kaffe, siger til børnene] Upsedasse, det var vist koffeinfri kaffe jeg fik givet jer. Skynd jer ud!
\act{Børnene prøver at kravle ud, pædagogerne er aggressive og holder fast i dem. Preben henter guitaren i desperation (stukket ud bag bagtæppe). Spiller en tone an og der falder ro over indhegningen. De går rundt og klapper børnene igen.}

\scene AntJ kommer ind på scenen fra bagtæppet.
\says{AntJ}[Helt roligt tonefald] Hr. Professor! Alarm, alarm. Biologerne er sluppet ud fra deres anlæg.
\says{AntJ}Helledusseda. Ja, det kan jeg da godt se, så må vi hellere gå ud og indfange dem! Vi får lige børnene i sikkerhed.

\scene Folk i biologtøj og -udstyr kommer ind i salen blandt publikum og nøgler, lupper. Nu kommer personale ind i salen for at indfange. Pædagoger og børn fordufter fra scenen.
\scene Mørkt på scenen

\scene Rødt lys på scenen. En personale stiller sig op på scenen.
\says{AntJ}[henvendt til publikum, forsøger at skjule sin ængstelse] Hallo derude! Ingen grund til panik \act{nervøs latter} men Biolog-situationen er lidt værre end først antaget; vi bliver nødt til at evakuere salen. Kom tilbage om 20 minutter, der burde vi have styr på det igen. Husk guldøl!

\scene Lyset i salen tændes

PAUSE


\end{sketch}
\end{document}