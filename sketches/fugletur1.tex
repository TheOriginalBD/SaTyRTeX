\documentclass[a4paper,12pt]{article}

\usepackage{revy}
\usepackage[utf8]{inputenc}
\usepackage[T1]{fontenc}
\usepackage[danish]{babel}
\pdfoptionpdfminorversion 4

\revyname{Biorevy}
\revyyear{2013}
% HUSK AT OPDATERE VERSIONSNUMMER
% UNDLAD AT SKRIVE I TEMPLATE.TEX - KOPIÉR OG OMDØB I STEDET FOR
\version{2}
\eta{$5$ minutter}
\status{Færdig}

\title{Filosofisk fugletur 1}
\author{Markus D.}

\responsible{<Freja>}

\begin{document}
\maketitle


%\begin{texxers}
%	\texxer{Navn}[email@exmpl.com]
%\end{texxers}


\begin{roles}
	\role{FK1}[Sidsel] Fuglekigger 1
	\role{FK2}[Freja] Fuglekigger 2
	\role{FK3}[aDA!] Fuglekigger 3
	\role{Fil}[Ejnar] Filosof (''Claus Emmeche'')
\end{roles}


\begin{props}
	\prop{Onitolog-ting}[Hvem skaffer?] Ornitolog-gear
	\prop{Onitolog-ting}[Hvem skaffer?] Ornitolog-gear
	\prop{Onitolog-ting}[Hvem skaffer?] Ornitolog-gear
	\prop{3 kikkerter} [Vi har dem ikke]
	\prop{tænkepibe/humanistpibe} []
	
\end{props}

\begin{sketch}

\scene \textbf{Bemærkninger til Teknikken:}
Sketchen foregår i en skov. Fruglekvidren og skovbelysning (grønt)


\scene lys op på tom scene. Fuglekiggerne og filosoffen kommer ind på scenen og ankommer til deres fuglekiggersted. Stemningen er god.

\says{Fil}[lystig og glad]Velkommen til dette års filosofiske fuglekigger-tur. Mit navn er Claus Emmeche, og som jeres videnskabsteoriprofessor vil jeg bistå jer i filosofiske betragtninger, når vi udforsker det vilde dyreliv.

\says{FK1}[Glad!]Se! Se derovre! En bogfinke!

\scene Fuglekiggere kigger over mod bandscenen med iver og kikkerter.

\says{Fil}[alvorlig tone]Nej, stop, stop. Der må jeg afbryde dig\ldots

\says{FK3}Hvad mener du?

\scene Alle kigger på Fil.

\says{Fil}Ja, det er jo ikke helt korrekt det, jeg hører dig sige.

\says{FK2}Ahhr! Den sidder altså lige derovre -- blåt hoved og halespids. 
\says {FK1}Det \emph{er} en bogfinke!

\scene FK'ere nikker i enighed. FK1 Peger på sin ''Nordens fugle''-bog.

\says{Fil}\emph{DER SIDDER IKKE NOGET DEROVRE}. Ifølge Kuhns inkommensurabilitets-begreb eksisterer det omtalte genstandsfelt uafhængigt af verden, og \emph{DU} konstruerer din egen subjektive erkendelse\ldots \act{kunstpause}

\says{FK1}Men jeg kan jo --

\says{Fil}[holder sig for ørene og råber]\emph{DU} skal ikke tvinge mig ind i \emph{DIN} virkelighedskonstruktion! \act{Pause -- så roligere} Jeg må \emph{insistere} på at denne fugletur foregår i videnskabsteoretisk korrekthed.

\scene Lang pause. FK1 tænker sig godt om.

\says{FK1+FK2+FK3}Undskyld Claus\ldots

\says{Fil}Ja tak.

\says{FK1}[rømmer sig]Ud fra min egen selvbevidsthed erkender jeg, at jeg har konstrueret en abstraktion der ud fra fuldstændig subjektive observationer bærer sammenligning til andre menneskers beskrevne konstruktioner om det genstandsfelt, der i triviel tale benævnes "Bogfinke".

\says{Fil}[lystig og munter igen]Nåhr hvor dejligt, hvor er den da bare flot.

Lys ned.

\end{sketch}
\end{document}