\documentclass[a4paper,12pt]{article}

\usepackage{revy}
\usepackage[utf8]{inputenc}
\usepackage[T1]{fontenc}
\usepackage[danish]{babel}
\pdfoptionpdfminorversion 4

\revyname{Biorevy}
\revyyear{2013}
% HUSK AT OPDATERE VERSIONSNUMMER
% UNDLAD AT SKRIVE I TEMPLATE.TEX - KOPIÉR OG OMDØB I STEDET FOR
\version{0}
\eta{$3$ minutter}
\status{Færdig}

\title{Filosofisk fugletur 2}
\author{Markus D.}
\responsible{Freja}
\begin{document}
\maketitle


%\begin{texxers}
%	\texxer{Navn}[email@exmpl.com]
%\end{texxers}


\begin{roles}
	\role{FK1}[Sidsel] Fuglekigger 1
	\role{FK2}[Freja] Fuglekigger 2
	\role{FK3}[aDA!] Fuglekigger 3
	\role{Fil}[Ejnar] Filosof (''Claus Emmeche'')
\end{roles}


\begin{props}
	\prop{Onitolog-ting}[Hvem skaffer?] Ornitolog-gear
	\prop{Onitolog-ting}[Hvem skaffer?] Ornitolog-gear
	\prop{Onitolog-ting}[Hvem skaffer?] Ornitolog-gear
	\prop{A/V ørneskrig.}[Hvem skaffer?] A/V ørneskrig.
	\prop{3 kikkerter} [Vi har dem ikke]
	\prop{tænkepibe/humanistpibe} []
\end{props}

\begin{sketch}



\scene \textbf{Bemærkninger til Teknikken:}
Er i mellemgangen kommer ind af højre (set fra scenen) bagindgang og ender ved venste (også set fra scenen) mellemgangs-indgang
Sketchen foregår i en skov. Fruglekvidren og skovbelysning (grønt)



\scene Fuglekiggerne og filosof Claus Emmeche står og kigger på fugle. 

\says{Fil}Ahh, en dejlig dag at kigge på fugle på. Og i sandhed en frugtbar dag for videnskabsteorien.

\says{AV}Et stort ørneskrig høres på A/V. Que: når Sidsel når hen til trappen ved venstre mellemgangsdør (set fra scenen)

\scene Fuglekiggere kigger rundt over det hele. En stor havørn er landet i nærheden.

\scene Kigger på fuglekiggerne, da nervøst prøver at få fremstammet de korrekte videnskabsteoretiske begreber. De er meget ængstelige.

\says{FK1}Øhhh jaaa, åhr [klør sig i håret], jeg erkender en observation -- 

\act{Fil}Afbryder.

\says{Fil}Nej.

\says{FK2}Jeg observerer et paradig-

\says{Fil}Nej!

\says{FK3}Ud fra empirisk dat-

\says{Fil}Neeeej!

\act{FK'ere kigger på hinanden og forsøger forgæves at komme på det rigtige.}

\says{FK1}Øh\ldots øh, altså jeg antager en realismediskussio-

\says{Fil}Nej.

\says{FK2}[Meget frustreret]DER ER EN MEGA SJÆLDEN HAVØRN LIGE DER!!

\says{AV}Lydeffekt af havørn der skriger og flyver væk.

\scene Alle er stille og overraskede.

\says{Fil}Ja, nu skræmte du den jo væk! [slår sig for panden].



\end{sketch}
\end{document}