\documentclass[a4paper,12pt]{article}

\usepackage{revy}
\usepackage[utf8]{inputenc}
\usepackage[T1]{fontenc}
\usepackage[danish]{babel}
\pdfoptionpdfminorversion 4

\revyname{Biorevy}
\revyyear{2013}
% HUSK AT OPDATERE VERSIONSNUMMER
% UNDLAD AT SKRIVE I TEMPLATE.TEX - KOPIÉR OG OMDØB I STEDET FOR
\version{1}
\eta{$5$ minutter}
\status{Færdig}

\title{Filosofisk fugletur 3}
\author{Markus D.}

\begin{document}
\maketitle


%\begin{texxers}
%	\texxer{Navn}[email@exmpl.com]
%\end{texxers}


\begin{roles}
	\role{FK1}[Sidsel] Fuglekigger 1
	\role{FK2}[Freja] Fuglekigger 2
	\role{FK3}[aDA!] Fuglekigger 3
	\role{Fil}[Ejnar] Filosof (''Claus Emmeche'')
\end{roles}


\begin{props}
	\prop{Onitolog-ting}[Hvem skaffer?] Ornitolog-gear
	\prop{Onitolog-ting}[Hvem skaffer?] Ornitolog-gear
	\prop{Onitolog-ting}[Hvem skaffer?] Ornitolog-gear
	\prop{Bjørnebrøl}[Hvem skaffer?] Bjørnebrøl.
	\prop{Bjørn}[Hvem skaffer?] Bjørnekostume
	\prop{3 kikkerter} [Vi har dem ikke]
	\prop{tænkepibe/humanistpibe} []
\end{props}

\begin{sketch}



\scene \textbf{Bemærkninger til Teknikken:}
Sketchen foregår i en skov. Fruglekvidren og skovbelysning (grønt)


\scene Fil. Claus Emmeche og fuglekiggere går rundt i skoven efter en lang dag.

\says{Fil}Nå, ja så er vi ved at nå til vejs ende med dagens filosofiske fugletur. Og sikke da en dag! 

\says{AV} Bjørnebrøl høres på A/V.

\says{FK1}Claus, der står en bjørn lige bag ved dig!

\says{Fil}Altså, jeg vil snart ikke finde mig i mere af jeres uvidenhed! I må simpelthen respektere god videnskabsteoretisk praksis!

\scene Bjørn kommer ind på scenen og brøler.

\scene Fil står ubemærket. 

\says{FK3}Claus, pas på! Den angriber dig!

\says{Fil}Interessant tilgang til reduktionismen -- men jeg kan virkelig ikke erkende det paradigme du forudsætter da det --

\scene Bjørn angriber Claus og bider ham i foden. Claus falder men snakker videre. Fuglekiggere gemmer sig bag tæppet kun med hovedet fremme og kigger i kikkerten.

\says{Fil}- ikke kan forklares via epistemologisk data, men blot ud fra din egen opfattelse af kausalitet --

\scene Bjørn flår benet af Claus (kan vi lave det?). Ellers en masse blod?

\says{Fil}-- for i din videnskab vil genstandsfeltet ''bjørn'' ikke indgå i\ldots en\ldots  \act{taler langsommere pga. blodtab}\ldots  realismediskussion med paradigmet \ldots \act{langsommere} subjektiv\ldots observation\ldots Videnskabsteorien\ldots  sejrer\ldots  derfor\ldots altid\ldots 

\scene Claus Emmeche trækkes ud af scenen af det store rovdyr og gisper sine sidste ord.

\scene Fuglekiggere går langsomt ind på scenen, der hvor Claus døde.

\scene De kigger på hinanden, og peger op på bandscenen.

\says{FK3}[Glad!] Se, en bogfinke!

\scene Lys ned


\end{sketch}
\end{document}