\documentclass[a4paper,11pt]{article}

\usepackage{revy}
\usepackage[utf8]{inputenc}
\usepackage[T1]{fontenc}
\usepackage[danish]{babel}


\revyname{DIKUrevy}
\revyyear{2013}
\version{3.1}
\eta{$3$ minutter}
\status{Klar til nazificering}

\title{Unikke kvindelige problemer}
\author{Troels, Mark, Andreas, Simon, Marcus og Mikkel}
\responsible{Magnus}

\begin{document}
\maketitle

\begin{roles}
    \role{L}[Asger] Instruktor
    \role{D0}[NB] Datalog
    \role{D1}[Magnus] Datalog
    \role{K}[Julie] Kvindelig datalog
\end{roles}

\begin{props}
\prop{Kvindeplanche}(Planche af en kvinde, i menneske højde.)
\end{props}

\begin{sketch}
  \says{D0} Det er et velkendt faktum, at kvindekønnet er voldsomt
  underrepræsenteret på DIKU!

  \says{D1} Men DIKU er jo desværre også blevet en del af den moderne
  tid...

  \says{D0} ...meget mod ledelsens ønsker...

  \says{D1} ...og for at undgå, at der presses hårde kønskvoter ned
  over os, sådan som det sker for mange virksomhedsbestyrelser, så må
  vi selv tage affære...

  \says{D0} ...og altså forøge antallet af kvindelige studerende.

  \says{D0}[peger på D1] Niels...

  \says{D1}[peger på D0] ...og Mathias...

  \scene{D0 og D1 bliver mere og mere ophidsede.}

  \says{D0} ...er med al ydmyghed DIKUs fremmeste kvindeeksperter!
  Gennem mange års intense studier af kvinden...

  \says{D1} ...på afstand...

  \says{D0} ...gennem glas...

  \says{D1} ...både kiggert og vindue...

  \says{D0} ...for ej at forglemme spionkamera...

  \says{D1} ...intensivt...

  \says{D0} ...timevis...

  \says{D1} ...i skraldespanden...

  \says{D0} ... på pigetoilettet!

  \scene{D0 og D1 tager sig selv i hvad de siger, og genvinder
    fatningen.}

  \says{D1} Men hvad er så problemet?  Hvorfor er der så få piger på
  DIKU?

  \says{D0} Ja, det må jo være fordi pigerne, i kraft af deres køn,
  oplever nogle unikke problemstillinger, som vi, som rigtige
  dataloger, ikke kender til.

  \says{D1} Derfor har vi valgt at tage ind og overvåge en ganske
  almindelig øvelsestime for en nystartet, kvindelig rus.

  \says{D0} Ja, vi vil være opmærksomme, og gribe ind, når
  instruktoren ikke tager hånd om de særlige, kvindespecifikke
  situationer, som der måtte opstå.

  \scene{Nu er det en øvelsestime.  L og K er til øvelse.  K sidder og
    arbejder ved sin datamat og med en SML-bog.}

  \says{K}[rækker hånden op] Undskyld, hvad var det nu genvejstasten
  var til at gemme i Emacs?

  \says{L} C-...

  \says{D0}[afbryder L og tager ham hen i den anden side af scenen]
  Stop lige... så hårde og faktuelle svar kan kvindesindet slet ikke
  håndtere.

  \says{L} Øh...

  \says{D1} Når en pige spørger dig om hvordan man gør noget, så er
  det et tegn på utryghed.  Gør det for hende!

  \scene{L bliver skubbet over til K af D1, og ser sig lidt forvirret
    tilbage.}

  \says{L} Øh... det skal jeg nok klare for dig.

  \scene{L trykker C-x C-s på tastaturet og kigger usikkert hen på D0
    og D1.  D1 vifter hen mod K igen.}

  \says{L}[tøvende] Og, øh... hvis du skal gemme igen... så bare kald
  på mig.

  \scene{D0 og D1 giver thumbs-up.}

  \says{L}[prøver at koncentrere sig om at komme tilbage til
  undervisningen] Boolske værdier er altid true eller false, og...

  \scene{D0 tager igen fat i L og trækker ham væk.}

  \says{D0} Nej, nej, nej...

  \says{D1} Sig mig, fatter du ikke piger?!

  \says{L} Øh...

  \says{D0} Boolsk logik?  \textit{Altid} sandt eller falsk?

  \says{D1} Det forstår de sig slet ikke på!  Lad være med at være så
  mandschauvinistisk!

  \says{D0}[rækker L et hæfte] Hér, lær hende om dette i stedet.

  \says{L} Øh, tak...\act{går tilbage til K og fortsætter
    undervisningen}.  Vi har ikke bare sandhedsværdier som "ja" og
  "nej", men også "måske", "det må du bestemme", "beklager", "ser min
  GUI for kantet ud i dette styresystem" og "ikke nu, jeg har
  hovedpine".

  \says{K} Men... hvad er konjunktionen af "`beklager"' og "`ikke nu, jeg har hovedpine"'?

  \says{L} Øh... det må vel være "`måske"'? \act{Kigger forvirret
    tilbage på D0 og D1, der bare trækker på skuldrene.}

  \says{K} Og hvordan så med de Morgans love?  Jeg er altså ikke
  sikker på at jeg forstår det...

  \scene{D0 og D1 bliver synligt oprevede.  De tager fat i L igen.}

  \says{D0} Vi er ved at miste hende!

  \says{D1} Hun kan slet ikke koncentrere sig!

  \says{D0} Det må være en hormonel ubalance... hun må være gået i
  gang med at menstruere!

  \says{D1} Ja!  Det gør piger hele tiden!

  \says{D0}[rækker L en tampon] Her, en tampon!  Skynd dig at
  give hende den!

  \says{L} Det vil jeg altså ikke.

  \says{D1} Afled hende!

  \says{D0}[løber over til K, tager fat i hende, og peger mod
  publikum]  Se, Torben Mogensen!

  \says{K} Orv, hvor?

  \scene{Mens K er afledt lægger D1 tamponen på hendes tastatur.}

  \says{K} Arj, hvad fanden er det?

  \scene{D0 og D1 kryber sammen i forventning.  L går i frustration.}

  \says{D0} Vi tænkte at...

  \says{D1} ...det kunne afhjælpe dit problem!

  \says{K} Mit eneste problem er at I to fjolser står og forstyrrer
  undervisningen!  Hvem er I?

  \says{D1} Hun får det værre!

  \says{D0} Kan det være veer?  Hun bliver jo irrationelt aggressiv!

  \says{D1} Vi må forsøge at gøre hende tryg!

  \says{D0} Men hvordan?!

  \says{D1}[til K] Øh... vil du med mig hjem?!

  \says{K} Hvad?!

  \says{D0}[til K] Nej, tag med \textit{mig} hjem!

  \says{D1}[til D0] Hun er min!

  \says{D0}[til D1] Jeg gav hende mest hjælp med lektierne!  Altså bør
  hun have sex med mig!

  \says{K} Det her er simpelthen for dumt. \act{Tager sin datamat og går.}

  \says{D1+D0} HVAD?!

  \says{D0} Efter alt vi har gjort for dig?

  \says{D1} Det er fandeme ikke fair!  Du er fandeme tarvelig!

  \says{K}[hånligt, offstage] Årh, lad nu være med at blive pigesur.
\end{sketch}

\end{document}
