\documentclass[a4paper,11pt]{article}

\usepackage{revy}
\usepackage[utf8]{inputenc}
\usepackage[T1]{fontenc}
\usepackage[danish]{babel}

\revyname{SaTyRrevy\texttrademark}
\revyyear{2014}
% HUSK AT OPDATERE VERSIONSNUMMER
\version{0.1}
\eta{$n$ minutter}
\status{Ikke færdig}
\responsible{Spectrum}

\title{Møbelsketch}

\begin{document}

  \maketitle

  \begin{roles}
      \role{S}[Sara] Stol
      \role{K}[Heine] Kost
      \role{T}[Valdbjørn] Tæppe
      \role{L}[Anna] Lampe
      \role{St}[Olivia] Stumtjener
      \role{sVO}[Magnus] StolVO
      \role{kVO}[Spectrum] KostVO
      \role{B}[Toke] Bord
      \role{N}[Julie] Ninja som kaster klapstolen ind
  \end{roles}

  \begin{props}
      \prop{AV: Love-tema fra ``Godfather''}[AV]
      \prop{AV: Klap-lyd}[AV]
  \end{props}


  \begin{sketch}

    \scene{Skuespillere står på scenen, klædt ud som møbler: (Toke) bord, (Anna) lampe, (Sara) lænestol.}
    \scene{Godfather tema lyder i baggrunden.}
    \scene{Lys op. Der skal være lys på hele scenen.}

    \says{S} Nogen skal dø! Og det skal ikke være mig.
    \scene{En kost og dens stumtjener kommer ind.}
    \says{K} Så mødes vi igen, Don Stoleone! Jeg kan se du kører den dyre Arne Jacobsen-stil.
    \says{S} Godt set, Signor Costanova! Ja, man skal jo være med på \emph{komoden}. Jeg vil give dig et tilbud du ikke kan afslå.
    \says{K}[sagt so en kost]Såå? Hvad \emph{kost}er det?
    \says{S}Ikke den slags tilbud.
    \says{K}Nå. Men kan jeg \emph{stole} på dig? Jeg husker episoden fra Kinagrillen, hvor du lå lige lovlig længe og lurede på \emph{divaen}. Du skulle \emph{skamle }dig!
    \says{S}Før du bliver for strid en \emph{børste}, bør du nok lige \emph{feje} for egen dør.
    \says{K}Bare fordi du \emph{sidder} godt i det, har du ikke ret til at \emph{koste} rundt med mig!
    \says{S}Din ven siger ikke så meget, hva’? Han er lidt stiv i det.
    \says{K}Nu synes jeg du \emph{skaber} dig.
    \says{S}\emph{Reolig} nu. Ja, jeg ved du ynder at \emph{moppe} folk, og du kommer godt ud i \emph{krogene}.
    \says{K}Det er også temmelig \emph{skuffende} med \emph{Sengeløse}-sagen.
    \says{S}Det er ikke mit \emph{bord}.
    \says{K}Aha! Hvorfor sidder du så ved det?
    \says{S}Nu skal du ikke \emph{borde} i det.
    \says{K}Nej. Men så tager jeg da gerne imod tilbudet.
    \says{S}\emph{Sofa}, so good, men nu må der ske noget nyt.
    \scene{Stumtjeneren trækker en kniv og bevæger sig hen mod Kosten.}
    \says{S}Pas på, han har knif!
    \scene{Lampen vælter og dræber stumtjeneren.}
    \says{St}[sagt som en dør] Ah, jeg \emph{dør}.
    \says{S}Der fik han en på \emph{lampen}.
    \says{K}[Opgivende] SUK! -- Jeg går i Tæppeland!
    \says{S}Ja, du må hellere \emph{skrubbe} af.
    \scene{Kosten forlader scenen. En klapstol kommer flyvende ind på scenen.}
    \scene{Kosten stikker hovedet ind.}
    \says{K}Klapstol?
    \scene{Lænestolen klapper i hænderne.} 
    \scene{AV: Klap-lyd afspilles mens stolen klapper.}
    \scene{Kosten forlader igen scenen.}
    \scene{Kort pause}.
    \scene{Kosten kommer ind igen}
    \says{K}Skal du have en forårsrulle med? \scene{Laaang pause...}
    \says{S}Lad gå. Men først skal tæppet gå!
    \scene{Tæppet der ligger på gulvet (en mand i dynebetræk) går ud.}

    \scene{Lys ned.}

  \end{sketch}
\end{document}
